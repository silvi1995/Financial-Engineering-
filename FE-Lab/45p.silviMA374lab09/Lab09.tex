%%%%%%%%%%%%%%%%%%%%%%%%%%%%%%%%%%%%%%%%%
% Programming/Coding Assignment
% LaTeX Template
%
% This template has been downloaded from:
% http://www.latextemplates.com
%
% Original author:
% Ted Pavlic (http://www.tedpavlic.com)
%
% Note:
% The \lipsum[#] commands throughout this template generate dummy text
% to fill the template out. These commands should all be removed when 
% writing assignment content.
%
% This template uses a Perl script as an example snippet of code, most other
% languages are also usable. Configure them in the "CODE INCLUSION 
% CONFIGURATION" section.
%
%%%%%%%%%%%%%%%%%%%%%%%%%%%%%%%%%%%%%%%%%

%----------------------------------------------------------------------------------------
%	PACKAGES AND OTHER DOCUMENT CONFIGURATIONS
%----------------------------------------------------------------------------------------

\documentclass{article}

\usepackage{fancyhdr} % Required for custom headers
\usepackage{lastpage} % Required to determine the last page for the footer
\usepackage{extramarks} % Required for headers and footers
\usepackage[usenames,dvipsnames]{color} % Required for custom colors
\usepackage{graphicx} % Required to insert images
\usepackage{listings} % Required for insertion of code
\usepackage{courier} % Required for the courier font
\usepackage{lipsum} % Used for inserting dummy 'Lorem ipsum' text into the template

% Margins
\topmargin=-0.45in
\evensidemargin=0in
\oddsidemargin=0in
\textwidth=6.5in
\textheight=9.0in
\headsep=0.25in

\linespread{1.1} % Line spacing

% Set up the header and footer
\pagestyle{fancy}
\lhead{\hmwkAuthorName} % Top left header
\chead{\hmwkClass\ (\hmwkClassInstructor\ \hmwkClassTime): \hmwkTitle} % Top center head
\rhead{\firstxmark} % Top right header
\lfoot{\lastxmark} % Bottom left footer
\cfoot{} % Bottom center footer
\rfoot{Page\ \thepage\ of\ \protect\pageref{LastPage}} % Bottom right footer
\renewcommand\headrulewidth{0.4pt} % Size of the header rule
\renewcommand\footrulewidth{0.4pt} % Size of the footer rule

\setlength\parindent{0pt} % Removes all indentation from paragraphs

%----------------------------------------------------------------------------------------
%	CODE INCLUSION CONFIGURATION
%----------------------------------------------------------------------------------------

\definecolor{MyDarkGreen}{rgb}{0.0,0.4,0.0} % This is the color used for comments
\lstloadlanguages{Perl} % Load Perl syntax for listings, for a list of other languages supported see: ftp://ftp.tex.ac.uk/tex-archive/macros/latex/contrib/listings/listings.pdf
\lstset{language=Perl, % Use Perl in this example
        frame=single, % Single frame around code
        basicstyle=\small\ttfamily, % Use small true type font
        keywordstyle=[1]\color{Blue}\bf, % Perl functions bold and blue
        keywordstyle=[2]\color{Purple}, % Perl function arguments purple
        keywordstyle=[3]\color{Blue}\underbar, % Custom functions underlined and blue
        identifierstyle=, % Nothing special about identifiers                                         
        commentstyle=\usefont{T1}{pcr}{m}{sl}\color{MyDarkGreen}\small, % Comments small dark green courier font
        stringstyle=\color{Purple}, % Strings are purple
        showstringspaces=false, % Don't put marks in string spaces
        tabsize=5, % 5 spaces per tab
        %
        % Put standard Perl functions not included in the default language here
        morekeywords={rand},
        %
        % Put Perl function parameters here
        morekeywords=[2]{on, off, interp},
        %
        % Put user defined functions here
        morekeywords=[3]{test},
       	%
        morecomment=[l][\color{Blue}]{...}, % Line continuation (...) like blue comment
        numbers=left, % Line numbers on left
        firstnumber=1, % Line numbers start with line 1
        numberstyle=\tiny\color{Blue}, % Line numbers are blue and small
        stepnumber=5 % Line numbers go in steps of 5
}

% Creates a new command to include a perl script, the first parameter is the filename of the script (without .pl), the second parameter is the caption
\newcommand{\perlscript}[2]{
\begin{itemize}
\item[]\lstinputlisting[caption=#2,label=#1]{#1.pl}
\end{itemize}
}

%----------------------------------------------------------------------------------------
%	DOCUMENT STRUCTURE COMMANDS
%	Skip this unless you know what you're doing
%----------------------------------------------------------------------------------------

% Header and footer for when a page split occurs within a problem environment
\newcommand{\enterProblemHeader}[1]{
\nobreak\extramarks{#1}{#1 continued on next page\ldots}\nobreak
\nobreak\extramarks{#1 (continued)}{#1 continued on next page\ldots}\nobreak
}

% Header and footer for when a page split occurs between problem environments
\newcommand{\exitProblemHeader}[1]{
\nobreak\extramarks{#1 (continued)}{#1 continued on next page\ldots}\nobreak
\nobreak\extramarks{#1}{}\nobreak
}

\setcounter{secnumdepth}{0} % Removes default section numbers
\newcounter{homeworkProblemCounter} % Creates a counter to keep track of the number of problems

\newcommand{\homeworkProblemName}{}
\newenvironment{homeworkProblem}[1][Problem \arabic{homeworkProblemCounter}]{ % Makes a new environment called homeworkProblem which takes 1 argument (custom name) but the default is "Problem #"
\stepcounter{homeworkProblemCounter} % Increase counter for number of problems
\renewcommand{\homeworkProblemName}{#1} % Assign \homeworkProblemName the name of the problem
\section{\homeworkProblemName} % Make a section in the document with the custom problem count
\enterProblemHeader{\homeworkProblemName} % Header and footer within the environment
}{
\exitProblemHeader{\homeworkProblemName} % Header and footer after the environment
}

\newcommand{\problemAnswer}[1]{ % Defines the problem answer command with the content as the only argument
\noindent\framebox[\columnwidth][c]{\begin{minipage}{0.98\columnwidth}#1\end{minipage}} % Makes the box around the problem answer and puts the content inside
}

\newcommand{\homeworkSectionName}{}
\newenvironment{homeworkSection}[1]{ % New environment for sections within homework problems, takes 1 argument - the name of the section
\renewcommand{\homeworkSectionName}{#1} % Assign \homeworkSectionName to the name of the section from the environment argument
\subsection{\homeworkSectionName} % Make a subsection with the custom name of the subsection
\enterProblemHeader{\homeworkProblemName\ [\homeworkSectionName]} % Header and footer within the environment
}{
\enterProblemHeader{\homeworkProblemName} % Header and footer after the environment
}

%----------------------------------------------------------------------------------------
%	NAME AND CLASS SECTION
%----------------------------------------------------------------------------------------

\newcommand{\hmwkTitle}{FE-Assignment\ \#09} % Assignment title
\newcommand{\hmwkDueDate}{Monday,\ April\ 4,\ 2016} % Due date
\newcommand{\hmwkClass}{MA\ 374} % Course/class
\newcommand{\hmwkClassTime}{} % Class/lecture time
\newcommand{\hmwkClassInstructor}{} % Teacher/lecturer
\newcommand{\hmwkAuthorName}{Silvi Pandey ( 130123045 )} % Your name

%----------------------------------------------------------------------------------------
%	TITLE PAGE
%----------------------------------------------------------------------------------------

\title{
\vspace{2in}
\textmd{\textbf{\hmwkClass:\ \hmwkTitle}}\\
\normalsize\vspace{0.1in}\small{Due\ on\ \hmwkDueDate}\\
\vspace{0.1in}\large{\textit{\hmwkClassInstructor\ \hmwkClassTime}}
\vspace{3in}
}

\author{\textbf{\hmwkAuthorName}}
\date{} % Insert date here if you want it to appear below your name

%----------------------------------------------------------------------------------------

\begin{document}

\maketitle
\newpage

\begin{center}
\textbf{PROBLEM 1}
\end{center}

Similar to the one done in Lab-08, collect the data of option prices on some of the stocks that are included in NIFTY
index for a time interval depending on the availability of data (going backwards from 8th March 2016). Choose the
stocks such that they are from different industries and are already included in your database “nsedata1”. The data
should comprise of closing prices of calls and puts of various maturities and strike prices. Put all these data in an
Excel file and name it as “stockoptiondata”.

\begin{center}
\textbf{SOLUTION}
\end{center}

The excel file has been included along with all the other work files.

\begin{center}
\textbf{PROBLEM 2}
\end{center}

Consider the data of option prices on NIFTY and on stocks stored in the Excel files “NIFTYoptiondata” and
“stockoptiondata”. Take the current time to be t = 0 and S 0 to be the current index level or the current stock price.
Assume r = 5\%.\\
1. Plot the option prices (for both call and put) for a range of maturities and strike prices in three dimension. (Your
plot axes are option price, maturity and strike price). If you visualize the above plot in two dimensions (option
price vs. strike and option price vs. maturity) what do you observe ?\\
2. For each maturity and each strike, compute the implied volatility from the BSM formula using the appropriate
root-finding method (eg. Newton-Raphson method).
Plot the implied volatilities against strike price and maturity in three dimension. What are your observations if
you examine the plot in two dimensions (implied volatility vs. strike and implied volatility vs. maturity) ?\\
3. Estimate the historical volatility for the same period for which you have estimated the implied volatility. How
do the two volatilities compare ? Present your results in tabular and graphical forms.
Note that when you are computing the historical volatilities, you have to take data starting from t = 0 and going
back in time for a period equal to the maturity of the option.\\

\begin{center}
\textbf{SOLUTION}
\end{center}

\textbf{Part 1}\\
\textbf{3D PlOTS}\\

\textbf{Plot of Maturity and Strike price Versus Call Option Price}
\begin{center}
\includegraphics[width =80mm]{Lab9_Q1-Figure1}
\end{center}

\textbf{Plot of Maturity and Strike price Versus Put Option Price}
\begin{center}
\includegraphics[width =80mm]{Lab9_Q1-Figure2}
\end{center}

\textbf{2D PlOTS}\\

\textbf{Plot of Strike price Versus Call Option Price}
\begin{center}
\includegraphics[width =80mm]{Lab9_Q1-Figure3}
\end{center}

\textbf{Plot of Maturity Versus Call Option Price}
\begin{center}
\includegraphics[width =80mm]{Lab9_Q1-Figure4}
\end{center}

\textbf{Plot of Strike price Versus Put Option Price}
\begin{center}
\includegraphics[width =80mm]{Lab9_Q1-Figure5}
\end{center}

\textbf{Plot of Maturity Versus Put Option Price}
\begin{center}
\includegraphics[width =80mm]{Lab9_Q1-Figure6}
\end{center}

\textbf{Part 2}\\

\textbf{Plot of Maturity and Strike Price Versus Implied volatility}
\begin{center}
\includegraphics[width =80mm]{Lab9_Q2-Figure1}
\end{center}

\textbf{Plot of Strike Price Versus Implied volatility}
\begin{center}
\includegraphics[width =80mm]{Lab9_Q2-Figure2}
\end{center}

\textbf{Plot of Maturity Versus Implied volatility}
\begin{center}
\includegraphics[width =80mm]{Lab9_Q2-Figure3}
\end{center}

\textbf{Part 3}\\

\textbf{Plot of Maturity and Strike Price Versus Historical volatility}
\begin{center}
\includegraphics[width =80mm]{Lab9_Q3-Figure1}
\end{center}

\textbf{Plot of Strike Price Versus Historical volatility}
\begin{center}
\includegraphics[width =80mm]{Lab9_Q3-Figure2}
\end{center}

\textbf{Plot of Maturity Versus Historical volatility}
\begin{center}
\includegraphics[width =80mm]{Lab9_Q3-Figure3}
\end{center}

From the above plots, it can be inferred that the historical and implied volatilities only match at some points (i.e a very small fraction of all the points at which the volatilities have been calculated ).

\end{document}